\chapter{Štatistické testovanie náhodnosti}
\label{chap:statistic-tests}

Náhodné dáta zohrávajú dôležitú úlohu v rôznych odvetviach informatiky. Veľkú rolu majú napríklad v kryptografii, pretože jedným z požiadavkov na zašifrované dáta je aj nemožnosť zistiť, či sú výstupom zo šifrovacej funkcie, alebo len náhodná sekvencia. Aby sme vedeli rozoznať, či šifrovacia funkcia spĺňa toto kritérium, potrebujeme nástroj, ktorým keď preskúmame dáta zistíme, či sú náhodné alebo nie. Avšak zistiť, či sú dáta náhodné nie je vôbec jednoduché, pretože vlastnosť naozaj náhodného generátora je, že vygeneruje každú sekvenciu s rovnakou pravdepodobnosťou. To znamená, že každá sekvencia môže byť výstupom z náhodného generátora.

Najrozšírenejším nástrojom na testovanie náhodnosti sú tzv. štatistické sady často nazývané aj štatistické batérie. Každá sada pozostáva z viacerých testov, kde každý test skúma požadovanú vlastnosť na vstupných dátach. Z každého spusteného testu je výstupom výsledok, všetky tieto výsledky sú následne skombinované do jedného konečného výsledku. Keďže sa nikdy nedá určiť na 100 per cent, či sú dáta naozaj náhodné, tak výsledok nemôže byť formou áno respektíve nie. Preto sa výsledok vyjadruje iba ako pravdepodobnosť s akou by naozaj náhodný generátor vygeneroval menej náhodné dáta ako testované dáta \cite{nist-sts-interpretation-syso}. 

\section{NIST STS}
\label{sec:sts-nist}

Najznámejšia zo štatistických sád na štatistické testovanie náhodnosti \textit{Statistical Test Suite for Random and Pseudorandom Number Generators for Cryptographic Applications}~\parencite{nist-sts-documentation}, ktorá vznikla v národnom inštitúte štandardov a technológií (NIST), používa vo svojej testovacej sade množinu 15-tich testov, ktoré boli zostavené na základe predstavy o tom, ako by mala náhodná sekvencia vyzerať. Napríklad pre každú sekvenciu vygenerovanú náhodným generátorom platí, že pri zápise v binárnej sústave je na každej pozícií s rovnakou pravdepodobnosťou 1 alebo 0. Preto je malá pravdepodobnosť, že naozaj náhodná sekvencia bude napríklad obsahovať len samé jedničky, naopak z oveľa vyššou pravdepodobnosťou bude počet 1 a 0 približne rovnaký. Z tohoto dôvodu je jedným z testov, ktorý obsahujú štatistické sady aj test, ktorý testuje či je táto vlastnosť splnená. Niektoré z testov majú dokonca parametre vďaka ktorým sa v konečnom dôsledku nad dátami môže spustiť niekoľko násobne viac inštancií testov. Cez rozhranie na príkazovom riadku sa dá pred spustením nastaviť akákoľvek kombinácia testov, ktorú si želáme spustiť. Avšak interpretácia výsledkov nie je triviálna. Každý test sa spustí nad viacerými sekvenciami. Ako výsledok testu, pre každú sekvenciu je p-hodnota z intervalu $[0, 1]$, ktorá značí pravdepodobnosť, že by naozaj náhodný generátor vygeneroval menej náhodnú sekvenciu. Napríklad pre 1000 sekvencií dostaneme ku každému testu 1000 p-hodnôt. Na určenie výsledku sa používajú 2 metódy:
\begin{myItemize}
	\item \textbf{Uniformné rozloženie p-hodnôt po celom intervale $[0, 1]$}\\Na to aby sa dáta dali považovať za náhodné by mali byť všetky p-hodnoty rovnomerne rozdelené po celom intervale $[0, 1]$. To znamená, že keď si interval rozdelíme na 10 častí podľa hodnoty, teda prvá časť bude obsahovať p-hodnoty z intervalu $[0, 0.1]$, druh časť p-hodnoty z intervalu $[0.1, 0.2]$ atď. v každej časti by malo skončiť rovnaké množstvo p-hodnôt, v našom prípade približne 100.
	\item \textbf{Pomer úspešných testov ku všetkým testom}\\Ďalší spôsob na určovanie výsledku je pomer úspešných testov ku všetkým. Na určenie výsledku potrebujeme hodnotu hladiny významnosti ($\alpha$) a interval do ktorého musí pomer spadnúť (interval je vypočítaný na základe hodnoty $\alpha$). Každá z 1000 p-hodnôt získaná z jedného testu je porovnaná s hodnotou $\alpha$. Ak je hodnota menšia, test pre jednu sekvenciu neprešiel, ak je väčšia, test prešiel. Pomer sekvencií ktoré testami prešli, ku všetkým, by mal ležať vo vypočítanom intervale.
\end{myItemize}
Podľa článku \textit{On the Interpretation of Results from the NIST Statistical Test Suite}~\parencite{nist-sts-interpretation-syso} je vysoká pravdepodobnosť, až 80\%, že aj výstup z naozaj náhodného generátora neprejde niekoľkými testami. Autori spomenutého článku ukázali, že dáta sa dajú považovať za náhodné aj vtedy, ak neprejde (to znamená p-hodnota je menšia ako $\alpha$ = 1\%) 6 testov. Autori v článku si na testovanie zvolili množinu 188 testov.

\section{Dieharder}
\label{sec:dieharder}
Dieharder~\parencite{dieharder} je sada vytvorená Robertom G. Brownom na Duke univerzite za účelom zrýchliť a zjednodušiť spúšťanie testov tak, aby ich mohol rýchlo a jednoducho spustiť každý, kto potrebuje o svojich dátach zistiť, či sú náhodné. Testovacia sada je následníkom Diehard-u~\parencite{diehard}, avšak testy sú upravené a začlenené do rovnakej štruktúry. Taktiež sú do nej pridané testy z iných sád alebo od iných samostatných autorov. Vo verzii 3.31.1 z roku 2003 sa nachádza 31 testov, z toho 17 pochádza z pôvodnej sady diehard, 3 zo sady STS NIST (autori očakávajú, že raz bude obsahovať všetky testy) a zvyšných 11 z rôznych zdrojov, napríklad aj od samotného autora R. G. Browna.

\section{TestU01}
\label{sec:testu01}
Tvorcom Test-U01~\parencite{testu01} je  Pierre L’Ecuyer, ktorý pôsobí na univerzite v Montreale. Táto sada obsahuje množstvo testov, ktoré sa dajú spúšťať rôznymi spôsobmi, napríklad prostredníctvom batérií alebo samostatne. Testy sú implementované v jazyku ANSI C. Knižnica je rozdelená do viacerých modulov, popis modulov je k dispozícii v dokumentácii~\parencite{testu01-documentation}, jedným z nich sú aj testovacie batérie. Sada obsahuje viac batérii testov, kde každá batéria má svoje určenie.
\begin{myItemize}
	\item \textbf{Small crush}\\Vytvorená tak, aby čo najrýchlejšie poskytla výsledok, preto neobsahuje veľa testov. Slúži na testovanie generátorov náhodných sekvencií.
	\item \textbf{Crush}\\Narozdiel od Small crush obsahuje viac testov. Zaberie teda viac času, avšak ak všetky testy prejdú, môžme si byť istejší pravdivosťou výsledku. Takisto ako small crash slúži na testovanie generátorov.
	\item \textbf{Big crush}\\Ešte väčšia a pomalšia batéria ako crush a small crash. 
	\item \textbf{Alphabit}\\Primárne určená na hardware generátory, na vstupe môže brať aj jeden binárny súbor.
	\item \textbf{Rabbit}\\Takisto ako Alphabit môže mať ako vstup binárny súbor.
	\item \textbf{A ďalšie}\\Obsahuje ešte iné batérie, ktoré simulujú batérie spomenuté vyššie, napríklad PseudoDIEHARD, ktorá simuluje batériu DIEHARD~\parencite{diehard}. Alebo batéria FIPS\_140\_2, ktorá napodobňuje STS od NIST.	 
\end{myItemize}
