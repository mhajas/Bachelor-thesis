\chapter{Simulátor Java bytecodu}
\label{chap:eacirc-jvmsim}

V tejto kapitole si predstavíme nový typ uzlov ako náhradu za uzly predstavené v predchádzajúcej kapitole. Tieto uzly nevykonávajú iba jednoduchú funkcionalitu (napr. \textit{AND, OR, XOR}), ale dokážu vykonávať inštrukcie získané z programu, ktorý je napísaný v jazyku Java. Celý princíp EACircu zostáva rovnaký, jedinou zmenou je vykonávaná funkcia v rámci jednotlivých uzlov. Nové uzly sa dajú jednoducho kombinovať s bežnými uzlami, pretože majú rovnakú štruktúru. To znamená, že v rámci jedného obvodu sa môžu použiť bežné uzly, a zároveň aj nové uzly. Tieto uzly vznikli za účelom pomôcť genetike, aby mohla čo najjednoduchšie vytvoriť výsledný obvod, ktorý bude mať čo najlepšiu úspešnosť v rozlišovaní medzi náhodnými dátami a preverovanými dátami. Základná myšlienka je taká, že obvod, ktorý v uzloch vykonáva inštrukcie, ktoré pochádzajú z implementácie kryptografickej funkcie, by mal mať vyššiu šancu rozlíšiť medzi náhodnými dátami a výstupom z tejto funkcie ako EACirc s bežnými uzlami. Jedným z hlavných cieľov tejto bakalárskej práce je overiť, či je táto myšlienka pravdivá.

\section{Motivácia za použitím inštrukcií z jazyka Java}
\label{sec:java-bytecode}

Java je na jednu stranu vysoko úrovňový jazyk, čo znamená, že je jednoduchší na učenie a tým pádom aj rozšírenejší, no na druhú stranu je jednoduché z programu, ktorý je napísaný v jave, získať nízko úrovňový binárny kód, ktorý sa dá interpretovať. Vďaka jej rozšírenosti by mala byť samozrejmosťou dostupnosť implementácie množstva pseudo náhodných generátorov a šifrovacích, respektíve hašovacích funkcií. Z tohoto pohľadu sa Java javí ako veľmi dobrý jazyk pre použitie popísané v tejto kapitole.

\section{Java Virtual Machine}
\label{sec:jvm}

JVM~\parencite{JVM} je software, ktorý spája kód naprogramovaný v Jave a hardware konkrétneho počítača, na ktorom tento kód spúšťame. Vďaka nemu je Java multiplatformová, pretože Javu je možné spustiť všade tam, kde beží JVM. Samotný JVM má podobné princípy ako bežný procesor, teda napríklad obsahuje zásobník, registre a vie vykonávať konečnú množinu inštrukcií. Avšak narozdiel od hardvérového procesora, je JVM iba softvérové riešenie, ktoré ho napodobňuje.

Fungovanie JVM nie je nič zložitého, jednoducho obsahuje implementáciu všetkých inštrukcií a po načítaní súboru z inštrukciami, ktorý sa nazýva bytecode, vykonáva jednu inštrukciu po druhej. Avšak poznať detailné fungovanie JVM nie je pre účely tejto práce dôležité. Dôležité je vedieť, že aj náš JVM simulátor vykonáva inštrukcie z bytecodu, avšak s tým rozdielom, že vykonáva náhodný kus inštrukcií a vykonáva ich v rámci uzlov EACircu. Naša implementácia nepozná úplne všetky inštrukcie, ktoré obsahuje klasický JVM. Implementované sú zatiaľ iba tie inštrukcie, s ktorými sme sa stretli v niektorom z použitých bytecodov a boli v rámci nášho simulátora implementovateľné. Ďalší rozdiel je napríklad v používaní zásobníka, v klasickom JVM má každá funkcia svoj vlastný lokálny zásobník, ktorý sa používa napríklad na predávanie argumentov, lokálne premenné atď. Zatiaľ čo náš simulátor používa zásobník ako globálne úložisko, na ktoré sa ukladajú hodnoty počas celého jedného výpočtu, ktorý prislúcha jednému uzlu.  

\subsection{Bytecode}
\label{subsec:bytecode}

Bytecode je označenie pre súbor, ktorý je výstupom po skompilovaní Java kódu. Má presne danú štruktúru, ktorú vie vykonávať každý JVM. Je to v podstate množina funkcií, kde každá funkcia obsahuje niekoľko inštrukcií. Každý riadok s inštrukciou obsahuje jej číslo, názov, prípadne argumenty, ktoré počas vykonávania využije. Napríklad inštrukcia \textit{bipush} očakáva ako argument číslo, ktoré vloží na zásobník. Príklad bytecodu je k dispozícií v prílohe.

\section{Princíp fungovania emulácie}
\label{sec:jvm-principle}

Fungovanie JVM simulátora je proces, ktorý sa skladá z viacerých bodov, avšak nie je to súvislý proces, jedná sa iba o obsluhu JVM uzlov. Zvyšok EACircu funguje rovnako ako pri bežných uzloch. V tejto sekcii si prejdeme celý proces krok po kroku, od načítavania bytecodu až po vykonávanie konkrétnych inštrukcií a následný výpočet výsledku. 

\subsection{Načítavanie bytecodu}
\label{parsing-bytecode}

Na začiatku behu EACircu sa musí JVM simulátor nainicializovať, čoho súčasťou je načítavanie bytecodu zo súboru. Funkcie a inštrukcie z tohoto súboru sa budú používať počas celého jedného behu. Jeho názov je uložený v konfiguračnom súbore v elemente \textit{JVM\_FILENAME}. Každá funkcia je následne uložená v spojovanom zozname a má priradené unikátne číslo a jej prislúchajúce inštrukcie. Takýmto spôsobom sa postupne načíta celý bytecode. Ak sa pri načítavaní vyskytne chyba, napríklad neznáma inštrukcia alebo zlá štruktúra bytecodu, vypíše sa chybová hláška a vykonávanie programu skončí.

V bytecode existuje množstvo inštrukcií. Niektoré z inštrukcií vyžadujú na svoje vykonanie aj argumenty, preto sa v štruktúre nachádza aj možnosť uloženia až dvoch argumentov.

\subsection{JVM uzly a voľba parametrov}
\label{subsec:jvm-nodes}

Každý uzol v súčasnej implementácií EACircu obsahuje 4 Bytovú informáciu. JVM uzol využíva celé 4 Byty a to nasledovne: \vspace{0pt}

\begin{myItemize}
	\item 1. Byte: tu je uložené, že sa jedná o JVM uzol. Každý typ uzla ma svoju konštantu, ktorá určuje aká funkcia sa má v rámci uzla vykonať. Napríklad ak je na tejto pozícii hodnota 19, znamená to, že sa jedná o JVM uzol.
	\item 2. Byte: číslo funkcie, ktorej inštrukcie sa budú vykonávať,
	\item 3. Byte: číslo riadka, na ktorom sa nachádza inštrukcia, od ktorej sa začína výpočet,
	\item 4. Byte: počet inštrukcií, koľko sa má vykonať. To znamená, že sa vykonávajú inštrukcie od tej na riadku z parametru číslo 3 po inštrukciu na riadku, ktorý vznikne spočítaním 3. a 4. parametra. Toto obmedzenie však neplatí pri vykonávaní funkcie zavolanej špeciálnymi inštrukciami \textit{invoke}. Význam týchto inštrukcií si rozoberieme v \hyperref[subsec:emulating-ins]{podsekcii 3.3.3}.
\end{myItemize}
JVM simulátor predpokladá, že každý uzol, ktorý bude vykonávať je validný. To znamená funkcia s číslom v parametri 2 musí existovať aj so začiatočnou inštrukciou a dostatkom inštrukcií na vykonávanie podľa posledných dvoch parametrov. Preto bola do EACircu doplnená funkcia, ktorá sa volá v prípade, že voľba vyberie, že sa jedná o JVM uzol. Táto funkcia vyberá parametre náhodne, avšak berie pri tom ohľad na to, aby bol výsledný uzol validný.

\subsection{Vykonávanie inštrukcií} 
\label{subsec:emulating-ins}

Ak EACirc pri vykonávaní obvodu narazí na JVM uzol automaticky zavolá JVM simulátor. Pred zavolaním funkcie, ktorá vykonáva inštrukcie, sa najprv musia predať JVM simulátoru všetky vstupy. Predávanie vstupov prebieha tak, že sa všetky vstupné hodnoty, ktoré idú do uzla, vložia na zásobník, ktorý obsahuje JVM simulátor. Až následne sa zavolá funkcia, ktorá sa stará o spúšťanie správnych inštrukcií. EACirc jej preto predá všetky parametre, ktoré sa v uzle nachádzajú, teda začiatočný riadok a počet inštrukcií, ktoré má vykonať. Okrem zásobníka obsahuje simulátor aj štruktúru, ktorá určuje stav procesora, teda obsahuje informácie o tom, ktorá funkcia sa vykonáva a na ktorom riadku. Po zavolaní funkcie sa táto štruktúra naplní dátami z uzla a následne sa pomocou nej posúva na každú nasledujúcu inštrukciu, ktorá sa bude vykonávať.

Existujú však aj inštrukcie, po vykonaní ktorých sa nepokračuje bežnou cestou. Teda pokračovaním na inštrukciu, ktorá nasleduje bezprostredne za práve vykonávanou inštrukciou. K týmto inštrukciám patria napríklad inštrukcie začínajúce na \textit{if} teda napríklad: \textit{ifeq, ifne, ifge} alebo \textit{ifle}. Tieto inštrukcie slúžia na vetvenie programu, ich funkciou je overiť, či je splnená. V prípade, že simulátor narazí na takúto inštrukciu a zároveň je splnená kontrolovaná podmienka preskočí na konkrétnu inštrukciu, ktorá sa vyskytuje v rovnakej funkcii aká sa práve vykonáva. Číslo inštrukcie, na ktorú sa preskakuje sa nachádza v argumente inštrukcie. Ďalšia inštrukcia, ktorá skáče v rámci funkcie je \textit{goto}, avšak narozdiel od inštrukcií \textit{if} skáče automaticky, bez kontrolovania podmienky. Ďalšie špeciálne inštrukcie sú tie, ktorých názov začína na \textit{invoke}. Tieto slúžia na volanie inštrukcií z iných funkcií, ktoré obsahuje bytecode. Medzi inštrukcie \textit{invoke} patria: \textit{invokespecial, invokestatic, invokevirtual}. Po vykonaní všetkých inštrukcií v zavolanej funkcií sa pokračuje na ďalšiu inštrukciu, ktorá nasleduje po inštrukcii \textit{invoke}. Teda napríklad ak sa pri vykonávaní funkcie číslo 1 zavolá funkcia číslo 2 pomocou niektorej z inštrukcií \textit{invoke}, simulátor začne vykonávať všetky inštrukcie z funkcie číslo 2 a po ich vykonaní sa vráti naspäť do funkcie číslo 1 a pokračuje inštrukciou, ktorá nasleduje za inštrukciou \textit{invoke}.

\subsection{Výsledok uzlu}

Cieľom vykonávania inštrukcií je spracovať nejakým spôsobom všetky vstupy, teda hodnoty na zásobníku a vytvoriť z nich hodnotu, ktorá sa predá na výstup. Keďže sa vykonáva náhodný kus inštrukcií, nemôžeme sa spoliehať na to, že bude po ich vykonaní na zásobníku len jedna hodnota. Preto sa po vykonaní všetkých inštrukcií ako výsledok uzla berie \textit{XOR} všetkých hodnôt na zásobníku. 

\section{Problémy spojené s implementáciou}
\label{sec:problems}

Jeden z problémov, ktorý má JVM simulátor je, že v súčasnej implementácií nie je možné uložiť do uzlu viac ako 4 Byty informácie. Keďže prvý Byte je rezervovaný pre funkciu, pre účely JVM simulátora ostávajú len 3 Byty. Po rozdelení máme teda pre každý z troch parametrov rozsah [0-255], teda 1 Byte. Z toho vyplýva, že s týmto rozdelením dokážeme vykonať maximálne 255 inštrukcií, a zároveň začať maximálne na riadku 255. V bežnom prípade, teda keď neberieme do úvahy inštrukcie, ktoré preskakujú na inú než nasledujúcu inštrukciu, posledná inštrukcia, ktorú dokážeme vykonať je teda na riadku 510. Čo znamená, že akákoľvek inštrukcia za ňou nemôže byť vykonaná. Avšak funkcie v bytecode môžu mať niekoľko násobne viac inštrukcií. V súčasnej dobe sa však prerába celá genetika v rámci EACircu a v novej implementácii by mal tento problém zaniknúť, pretože by malo byť v uzle viac miesta pre parametre.

Keďže existujú inštrukcie vďaka ktorým sa nemusí pokračovať na bezprostredne nasledujúcu inštrukciu, ale môže sa preskočiť kamkoľvek v rámci vykonávanej funkcie, môže sa stať, že sa vykonávanie zacyklí. Napríklad v prípade, že sa pomocou inštrukcie \textit{goto} skáče niekam nad aktuálne vykonávanú inštrukciu. Následne sa na túto inštrukciu príde znova a znova sa preskočí na pôvodné miesto a toto sa opakuje donekonečna. Takáto situácia môže nastať kedykoľvek, najmä kvôli tomu, že vykonávame náhodný kus inštrukcií. Napríklad môže nastať situácia, kedy inštrukcia potrebuje niečo, čo poskytoval kód, ktorý sme preskočili. Preto sme potrebovali zamedziť tomu, aby takýmto spôsobom program pokračoval donekonečna. Ako riešenie postačilo limitovať počet vykonaných inštrukcií v rámci behu, ktorý prislúcha jednému uzlu. Tento limit bol nastavený na 300 inštrukcií.

Okrem pokračovania donekonečna môže nastať aj situácia, kedy sa program síce nezacyklí donekonečna, ale počet vykonávaných inštrukcií výrazne narastie. Táto situácia môže nastať napríklad kvôli inštrukciám \textit{invoke}, ktoré volajú iné funkcie v rámci bytecodu. Napríklad ak sa zavolá funkcia, ktorá má mnohonásobne viac inštrukcií ako 255, čo je maximum vykonaných inštrukcií v bežnom prípade. Keďže EACirc v rámci jedného behu vykonáva veľké množstvo uzlov, zavolanie takejto funkcie by mohlo nadmerne zvýšiť časovú náročnosť celkového behu EACircu. Preto aj v tomto prípade bolo potrebné nastaviť limit na 300 inštrukcií, v opačnom prípade by sa síce behy nepočítali donekonečna, ale medzi časmi jednotlivých behov by mohol byť výrazný rozdiel.

\section{Implementačné rozdiely medzi skutočným JVM a našim JVM simulátorom}
\label{sec:impl-diff}

Okrem rozdielu, že nevykonávame funkcie v bytecode ako celok, ale pre každý uzol vykonávame náhodný kus inštrukcií z náhodnej funkcie, je ďalším rozdielom neúplnosť implementácie inštrukcií v našom JVM simulátore. Okrem inštrukcií, ktoré JVM simulátor nepozná vôbec, existujú aj inštrukcie, ktoré síce pozná, ale nemá ich implementované. Dôvodom je, že nie všetky inštrukcie boli pre nás výhodné na implementáciu čo sa týka pomeru vynaloženého úsilia a pridanej hodnoty. Preto sa pri vykonávaní týchto inštrukcií nevykoná nič a tieto inštrukcie sa automaticky preskočia. Patria sem napríklad inštrukcie na prácu s poľami a objektami. Celý zoznam takýchto inštrukcií je vypísaný v prílohe. 

\section{Výhody prístupu}
\label{sec:advantages}

Najväčšou výhodou je možnosť používať inštrukcie priamo z generátora, ktorý vytvoril preverované dáta. Okrem toho je ďalšou výhodou aj to, že bytecode sa môže skladať zo zložitejších častí a vďaka tomu je možné vytvoriť naozaj komplexný obvod. Otázkou ale je, či je genetika natoľko silný nástroj, aby bola schopná zložiť takéto komplexné obvody, pretože s použitím JVM simulátora je naozaj mnohonásobne viac možností ako výsledný obvod poskladať. Preto je ďalšou otázkou aj to, či pre JVM obvody nie je potrebné viac času, a teda viac generácií na hľadanie výsledného obvodu. Odpovede aj na tieto otázky sa pokúsime nájsť pri experimentoch, ktoré obsahuje táto práca.

\section{Nevýhody JVM uzlov}
\label{sec:disadvantages}

Najväčšou nevýhodou je dĺžka výpočtu, avšak takáto dĺžka je očakávaná, pretože sa v uzloch vykonávajú časovo o dosť zložitejšie výpočty pri porovnaní s bežnými uzlami. Zatiaľ čo bežný uzol sa dá prirovnať k jednej vykonanej inštrukcii medzi všetkými vstupmi, JVM simulátor ich môže v rámci jedného uzlu vykonať až 300. S vykonávaním inštrukcií je spojená aj réžia, ako napríklad kontrola počtu hodnôt na zásobníku alebo kontrola prekročenia maximálneho počtu inštrukcií. Takže v konečnom dôsledku je jeden beh mnohonásobne dlhší, ako beh s bežnými uzlami.

Niektoré inštrukcie, napríklad \textit{iadd}, ktorá vyberie zo zásobníka dve čísla, spočíta ich a výsledok vloží na zásobník, vyžadujú niekoľko hodnôt na zásobníku, a nie vždy sa tam tieto hodnoty vyskytujú. Z tohoto dôvodu sa stáva, že sa inštrukcie musia preskočiť. To znamená, že ak je zásobník prázdny, žiadna inštrukcia vyžadujúca hodnoty na zásobníku sa nevykoná. Avšak tento nedostatok sa nedá vyriešiť jednoduchou cestou, pretože je spôsobený vykonávaním náhodných inštrukcií, mnohokrát aj zo stredu funkcie. Môže sa teda stať, že vo veľa uzloch sa nevykoná vôbec nič. Potencionálne však máme možnosť vytvoriť zložitý obvod aj keď niekoľko uzlov nerobí nič. Iný pohľad na vec je, že aj obvody, v ktorých je veľa takýchto uzlov, môžu mať v konečnom dôsledku vysokú úspešnosť čo sa týka rozoznávania náhodných dát, preto je len na genetike, aby si vybrala ten správny prístup. 
