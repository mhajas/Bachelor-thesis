\chapter*{Úvod}
\label{chap:introduction}
\addcontentsline{toc}{chapter}{\textbf{Úvod}}

Použitie kryptografie sa stalo v dnešnej dobe nevyhnutnou súčasťou bežného života. Každé spojenie, cez ktoré komunikujeme je nejakým spôsobom zabezpečené práve pomocou kryptografických funkcií. Aj najmenšia chyba v takýchto funkciách by mohla spôsobiť zneužitie súkromných dát a~tomu sa snaží množstvo odborníkov na celom svete zabrániť. Preto potrebujeme stále lepšie nástroje, ktoré dokážu odhaľovať nedostatky implementácie kryptografických funkcií.

Jednou z~požiadavok na kryptografické funkcie je aj nemožnosť zistiť, že dáta pochádzajú z~kryptografickej funkcie, pretože potencionálnemu útočníkovi by mohlo takéto zistenie pomôcť v prelomení zašifrovaných dát. Aby sa nedalo určiť, či dáta pochádzajú z~kryptografickej funkcie, mali by sa tieto dáta čo najviac podobať na náhodnú sekvenciu. Z~tohto dôvodu vznikli nástroje, ktoré zisťujú pravdepodobnosť, že sú skúmané dáta náhodné. V prvej kapitole si predstavíme nástroje, ktoré používajú veľké korporácie, ktoré určujú kryptografické štandardy na testovanie tejto vlastnosti. Jedná sa o štatistické sady testov na testovanie náhodnosti. Predstavíme si 3 konkrétne najpoužívanejšie štatistické sady a~síce {STS NIST}~\parencite{nist-sts-documentation}, Dieharder~\parencite{dieharder} a~TestU01~\parencite{testu01}. 

Z dôvodu niektorých nedostatkov štatistických testov vznikol k štatistickým sadám alternatívny prístup, ktorý si predstavíme v druhej kapitole. Jedná sa o framework EACirc, ktorý používa na testovanie náhodnosti funkcie, ktoré rozlišujú medzi náhodnými dátami a~výstupom z~kryptografickej funkcie. Takáto funkcia je EACircom vytváraná automaticky. Na vytváranie sa používa samovzdelávací genetický algoritmus. Výhodou EACircu je, že tieto testy sa vytvárajú takmer bez potreby ľudského zásahu, zatiaľ čo testy zo štatistických sád sú väčšinou zložité funkcie, nad ktorými museli ľudia stráviť množstvo času.

Táto práca bola vytvorená za účelom rozšíriť framework EACirc o simulátor Java bytecodu. V tretej kapitole si preto vysvetlíme čo je to Java bytecode a~takisto ako sa takýto bytecode vykonáva pomocou vytvoreného JVM simulátora. Ďalej si ukážeme ako je tento JVM simulátor prepojený s implementáciou EACircu. Požiadavka na toto rozšírenie bola, aby sa v rámci EACircu mohli vykonávať náhodné kusy inštrukcií, ktoré pochádzajú z~programu, ktorý je napísaný v Jave. 

V poslednej kapitole nájdeme popis a~výsledky experimentov, ktorými bola otestovaná implementácia EACircu s novým rozšírením a~to s použitím JVM simulátora. Takisto tam nájdeme zhrnutie jednotlivých výsledkov a~porovnanie s výsledkami zo štandardného EACircu. Postupne si predstavíme 3 experimenty.

\begin{myItemize}
	\item Prvý z~nich bol zameraný na otestovanie funkčnosti implementácie EACircu, ktorá obsahuje JVM simulátor.
	\item Druhý experiment bol zameraný na zistenie toho, či má EACirc, ktorý v sebe používa inštrukcie z~nejakej konkrétnej kryptografickej funkcie, väčšiu šancu na rozlíšenie výstupu z~tejto funkcie ako EACirc s bežnými uzlami.
	\item A nakoniec posledný experiment, v ktorom sa budeme snažiť zistiť, či je genetika schopná poskladať hľadanú funkciu aj zo zložitejšieho bytecodu.
\end{myItemize}

EACirc je tímový projekt, na ktorom sa pracuje v rámci laboratória CRoCS~\cite{crocs} (Centre for Research on Cryptography and Security) na Fakulte informatiky Masarykovej univerzity. Preto sú niektoré súčasti tejto práce výsledkom tímovej spolupráce. Úpravy implementácie a~experimenty boli realizované vo väčšej miere mnou, avšak po dohovore s ostatnými členmi tímu. Mojou úlohou bolo doplniť už existujúcu implementáciu JVM simulátora tak, aby sa dala táto implementácia používať v rámci frameworku EACirc. Taktiež som autorom všetkých výpočtov, ktoré testovali tento nový prístup.

Zdrojový kód frameworku EACirc je voľne dostupný na GitHube~\parencite{eacirc-github}. Implementácia, ktorú vysvetľuje táto práca je dostupná vo vetve, ktorá ma názov jvmsim. Stránky obsahujú aj prehľadnú dokumentáciu, kde sa dá nájsť viac informácií v prípade nejasností. 

Text práce bol vytvorený v sádzacom systéme \LaTeX{} s využitím balíka \textit{fithesis2}~\cite{fithesis2}.
