\chapter{Experimenty}
\label{chap:experiments}

Hlavným cieľom zavedenia JVM uzlov bolo vylepšiť úspešnosť EACircu. Na otestovanie som preto musel vyskúšať viacero experimentov, ktoré si predstavíme v tejto kapitole a porovnáme ich výsledky s výsledkami, ktoré dosiahol EACirc s bežnými uzlami pri rovnakých nastaveniach. 

Pri porovnávaní s bežnými uzlami sú používané výsledky z Baseline experimentu \cite{baseline-experiment}. 

\section{Experiment s imitáciou bežných uzlov}
\label{sec:exp1}

Prvý experiment slúžil najmä na overenie, či JVM simulátor funguje korektne. Myšlienkou bolo na začiatok nepoužívať bytecode vytvorený z implementácie šifrovacej funkcie, ale skúsiť vytvoriť bytecode, ktorý bude napodobňovať bežné uzly, ktoré sa nachádzajú v EACircu. Očakávané boli podobné výsledky ako dosiahol štandardný EACirc.

\subsection{Použitý bytecode}
\label{subsec:exp1-bytecode}

Bytecode pre tento experiment som musel napísať ručne, pretože si to vyžadovalo zamyslenie sa nad tým, ako funguje každá jedna funkcia z bežných uzlov. Nie pri všetkých funkciách to bolo jednoduché, pretože prístup JVM simulátora je odlišný od prístupu bežných uzlov. 

Jednou z vecí, ktorú je zložité napodobniť v prípade JVM uzlov je, že bežné uzly vykonávajú medzi všetkými vstupmi operáciu, napríklad \textit{AND, OR, XOR} atď. Teda spracujú všetky vstupy, ktoré takýto uzol dostane. V prípade JVM simulátora nie je problém nájsť inštrukcie, ktoré plnia ekvivalentnú funkciu ako bežné uzly. Problémom je, že bežné uzly vykonajú operáciu vždy medzi všetkými vstupmi, zatiaľ čo JVM simulátor vykonáva v jednom uzle náhodné množstvo inštrukcií. Preto bolo potrebné premyslieť, ako ich vykonať dostatok na to, aby sa zvolená inštrukcia vykonala medzi všetkými vstupmi, teda medzi všetkými hodnotami na zásobníku. Tento problém by sa dal vyriešiť tým, že by sme vykonávali vždy všetky inštrukcie, ktoré funkcia obsahuje, pretože by sme mohli presne určiť, koľko inštrukcií sa má vykonať. Avšak toto riešenie by bolo v rozpore so základnou myšlienkou JVM uzlov, a tou je vykonávať náhodný kus inštrukcií, ktoré pochádzajú zo šifrovacej funkcie. Je zrejmé, že ak budeme vykonávať vždy náhodnú časť inštrukcií, nie v každom prípade dosiahneme požadovanú funkcionalitu. Zvolili sme preto možnosť, že pre funkcie, ktoré potrebujú spracovať všetky hodnoty zo zásobníka, aspoň zvýšime pravdepodobnosť, že sa vybraných inštrukcií vykoná dostatok. Preto každá takáto funkcia obsahuje viackrát pod sebou vybranú inštrukciu tak, aby genetika mala pri náhodnom vyberaní inštrukcií väčšiu šancu zvoliť presne taký počet inštrukcií, aký je treba na spracovanie všetkých hodnôt na zásobníku. Inštrukciu sme vybrali podľa toho aby plnila rovnakú funkcionalitu, napríklad v prípade funkcie \textit{AND} je to inštrukcia \textit{iand}, ktorá vyberie zo zásobníka dve čísla, vykoná medzi nimi operáciu \textit{AND} a výsledok vloží naspať na zásobník. 

Niektoré funkcie sa nedajú nahradiť jedinou ekvivalentnou inštrukciou, napríklad \textit{ROTL} alebo \textit{ROTR}, ktoré vykonávajú pravú respektíve ľavú rotáciu. Ich implementácia obsahuje viac rôznych inštrukcií. V tomto prípade takisto nemôžeme garantovať, že sa vykoná celá funkcia preto, lebo začíname emuláciu od náhodného riadka. Podobne ako v predchádzajúcom prípade, ani tu sme nepoužili vykonávanie vždy celej funkcie, kvôli rozporu so základnou myšlienkou JVM uzlov.

Ďalší problém, ktorého výskyt sa ešte znásobil pri použití funkcií, ktoré vykonávajú náhodné množstvo rovnakých inštrukcií je, že tieto inštrukcie potrebujú na zásobníku minimálne toľko hodnôt, aký je ich počet. V opačnom prípade sa snažia vyberať hodnoty z prázdneho zásobníka. Preto sme sa museli zamyslieť aj nad otázkou, ako sa zachovať, ak je na zásobníku nedostatočný počet hodnôt. Z tohoto dôvodu som musel upraviť implementáciu tak, aby vyberanie hodnôt s prázdneho zásobníka nebol problém a nijak neovplyvňoval doterajší výpočet. Do úvahy pripadali možnosti z nasledujúceho zoznamu.

\begin{myItemize}
 \item \textbf{Pri vyberaní hodnoty z prázdneho zásobníka vracať 0}\\Toto riešenie sa ukázalo ako nesprávne, napríklad v súvislosti s inštrukciou \textit{AND}. Súčasťou vykonania inštrukcie AND je výber dvoch hodnôt zo zásobníka. Avšak, vo chvíli keď sa na zásobníku nachádza len jedna hodnota, simulátor vyberie práve túto jednu hodnotu a za druhú, neexistujúcu hodnotu, dosadí vždy nulu. A keďže výsledok výpočtu \textit{AND-u}, v ktorom sa nachádza 0 je vždy 0, tak stratíme celý doterajší výpočet vždy, keď takáto situácia nastane.
 \item \textbf{V prípade prázdneho zásobníka vracať neutrálnu hodnotu}\\Tento spôsob by vyriešil problém predchádzajúceho riešenia, pretože v prípade, keď sa vyberá z prázdneho zásobníka by sa vracala taká hodnota, s použitím ktorej by sa operácia správala ako identita, teda by vracala vždy hodnotu prvého argumentu. Avšak nevýhodou je, že implementácia takéhoto riešenia by bola zložitejšia a taktiež je zbytočné vykonávať niečo, čo aj tak nebude mať žiadny dôsledok, keďže sa obsah zásobníka nezmení.
 \item \textbf{Preskakovať inštrukcie, ktoré nemajú dostatok hodnôt na zásobníku}\\Toto riešenie sme zvolili ako najlepšie z dvoch dôvodov. Prvý sa týka výkonu, nakoľko sa nepočítajú zbytočné výpočty a druhý sa týka vyriešenia pôvodného problému. 
\end{myItemize}

\subsection{Výsledky experimentu}
\label{subsec:exp1-results}

Takto vytvorený bytecode som použili na rozlíšenie medzi náhodnými dátami a hašovacou funkciou Tangle. Pre každý výpočet som spustil 1000 behov na jeden výpočet, teda výsledok v každej bunke je vypočítaný z 1000 behov a určuje percentuálny počet behov, ktoré boli úspešné. Výsledky boli vyhodnotené pomocou nástroja Oneclick~\cite{obratil-bc}. Sivé zafarbenie bunky znamená, že EACirc bol úspešný vo viac prípadoch ako je hladina významnosti, teda 5\%, to znamená číslo vo vnútri bunky je väčšie ako 0.05. Je otázne, či takto označovať aj bunky, ktoré majú tesne nad 5\%, napríklad 0.055 alebo 0.06, pretože sa môže jednať aj o štatistickú odchýlku. Preto som sa rozhodol, že takto budem označovať len bunky, ktoré budú mať hodnotu nad 0.07.
 
\vspace{1em}
 \begin{table}[h]
\begin{tabularx}{\textwidth}{|C||C|C|C|}
	\hline
	\multicolumn{4}{|c|}{\textbf{Výsledky pre hašovaciu funkciu Tangle}} \\
	\hline \hline
	\vspace{0.1em}
	\textbf{Počet rúnd} &
	\vspace{0.1em}
	\begin{tabular}[b]{@{}c}\large\textbf{Bežné uzly} \\ \scriptsize(30000 generácií) \end{tabular} &
	
	\vspace{0.1em}
	\begin{tabular}[b]{@{}c}\large\textbf{JVM simulátor} \\ \scriptsize(30000 generácií) \end{tabular} &
		\vspace{0.1em}
		\begin{tabular}[b]{@{}c}\large\textbf{JVM simulátor} \\ \scriptsize(300000 generácií) \end{tabular} \\
	\hline\hline
	21 & 0.944\cc & 0.270\cc & 0.461\cc \\
	\hline
	22 & 0.932\cc & 0.267\cc & 0.465\cc \\
	\hline
	23 & 0.066 & 0.036 & 0.050 \\
	\hline
	
\end{tabularx}
\caption{Výsledky pre hašovaciu funkciu Tangle pri použití JVM uzlov, ktoré napodobňujú bežné uzly.}
\label{tab:exp1}
\end{table}

Počet rúnd odráža silu kryptografickej funkcie. Čím viac rúnd, tým by mala byť funkcia silnejšia a dáta by mali vypadať náhodnejšie. Preto je očakávané, že pri niektorej runde už nebude EACirc úspešný v rozoznávaní medzi náhodnými dátami a výstupom z hašovacej funkcie Tangle. Rundy pre tento experiment som volil podľa výsledkov EACircu s bežnými uzlami, vybral som poslednú rundu kde je EACirc s bežnými uzlami úspešný a pridal som ešte predchádzajúcu a nasledujúcu.

Z výsledkov v \hyperref[tab:exp1]{tabuľke 4.1} vyplýva, že JVM simulátor síce dokázal rozoznať tie isté rundy ako bežné uzly, avšak s menšou istotou. To znamená, napríklad pre rundu 21, že zatiaľ čo bežné uzly boli schopné v 94\% behov nájsť hľadaný obvod, EACirc s JVM uzlami bol to isté schopný spraviť len v 27\% behov. Dôvodom menšej úspešnosti sú zrejme nevyriešené problémy spomenuté vyššie.

Pri výpočte na 300000 generáciách bolo EACircu poskytnuté 10 krát viac generácií na hľadanie obvodu, čiže sú očakávané lepšie výsledky. Avšak ani v tomto prípade nedosiahol EACirc s JVM uzlami lepšie výsledky ako mali bežné uzly. Preto som sa rozhodol pokračovať v testovaní ďalšieho experimentu, v ktorom už budú použité bytecody zo šifrovacích respektíve hašovacích funkcií.

\section{Experiment s bytecodom z kandidátnych funkcií SHA3 a eStream}
\label{sec:exp2}

V tomto experimente som plnohodnotne využil najväčšiu výhodu JVM uzlov a to použitie bytecodu z reálnych funkcií, konkrétne z hašovacích funkcií zo sútaže na funkciu SHA-3: \textit{Tangle, Dynamic SHA, Dynamic SHA-2} a z prúdovej šifrovacej funkcie z kandidátov na funkciu eStream: \textit{Decim}. Experiment spočíval v tom, že som z týchto všetkých funkcií vytvoril bytecode a spustil všetky kombinácie výpočtov, teda každú funkciu v kombinácii s každým bytecodom. Motivácia za týmto experimentom je zistiť, či bude EACirc úspešnejší, keď bude v uzloch pomocou JVM simulátora vykonávať inštrukcie zo skúmanej kryptografickej funkcie. 

\section{Použité bytecody}
\label{sec:exp2-bytecode}

Naneštastie sa mi nepodarilo nájsť implementáciu ani jednej z týchto funkcií v Jave, preto som musel vytvoriť vlastnú. Najjednoduchšie riešenie bolo prepísať implementáciu z C++, ktorá je dostupná napríklad aj priamo v EACircu, do Javy. K prepisu som použil voľnú verziu konvertora~\parencite{c++-java-converter} z C++ do Javy, od spoločnosti \textit{Tangible Software Solutions Inc}. Po konvertovaní som musel ešte manuálne upraviť Java súbor tak, aby bol kompilovateľný, pretože niektoré konštrukcie z C++ sa nedali automaticky skonvertovať do Javy, napríklad práca s pamäťou alebo smerníková aritmetika. Implementácia, ktorá vznikla týmito krokmi by zrejme nebola plne funkčná ani korektná, ale na naše účely, teda na využitie inštrukcií, je postačujúca. 

Takto vytvorenú implementáciu som potom jednoducho skompiloval a zo skompilovaného súboru získal konkrétne inštrukcie. Vzniknutý súbor s bytecodom bolo treba jemne upraviť, aby si s ním poradila funkcia na načítavanie bytecodu, ktorú obsahuje JVM simulátor. Išlo napríklad o vymazanie niektorých tabulátorov a podobne. Posledným krokom bolo implementovanie inštrukcií, ktoré JVM simulátor zatiaľ nepoznal, pretože inak by načítavanie zlyhalo.

\section{Výsledky experimentu}
\label{sec:exp2-results}

Pre každú funkciu, som spustil výpočty pre 3 rundy. Rundy som vyberal rovnako ako v predchádzajúcom experimente. Našiel som poslednú rundu, kde bol EACirc s bežnými uzlami úspešný a do mojich výpočtov zahrnul túto rundu plus predchádzajúcu a nasledujúcu. Pre každú rundu som spustil výpočty v kombinácií s každým bytecodom, dohromady teda 4 výpočty na každú rundu, pretože máme k dispozícii 4 bytecody. To znamená, že dohromady som spustil 12 rôznych výpočtov pre každú funkciu. Každý z 12-tich výpočtov som spustil 1000 krát, čo znamená, že dovedna som potreboval spustiť pre jednu funkciu 12000 behov EACircu. Výpočty som spúštal na metacentre~\parencite{metacentrum}, kde sa behy spúšťajú v rámci tzv. úloh. V rámci jednej úlohy zvládlo metacentrum vypočítať 8 behov za menej ako dva dni. Dĺžka výpočtu jednej úlohy závisela od stroja, na ktorom sa počítala. Niektoré zvládli všetkých 8 behov za 10 hodín, zatiaľ čo niektorým to trvalo aj 30 hodín. Keďže metacentrum naraz priraďuje až 800 virtuálnych strojov s jedným procesorom, všetky úlohy pre jednu funkciu sa stihli vypočítať za približne 2 až 3 dni.

Z každého prúdu som použil 1000 testovacích vektorov. Pre výsledky platí to isté čo v predchádzajúcom experimente a teda, že šedo podfarbené bunky znamenajú, že EACirc bol úspešný a našiel obvod vo viac ako 5\% prípadov, plus štatistická odchýlka. Výsledky som vyhodnocoval pomocou nástroja Oneclick.

\resultsTable
{Výsledky pre funkciu Tangle}
{
	21 & 0.944\cc & 0.605\cc & 0.609\cc & 0.643\cc & 0.453\cc \\
	\hline
	22 & 0.932\cc & 0.640\cc & 0.606\cc & 0.666\cc & 0.461\cc \\
	\hline
	23 & 0.066 & 0.045 & 0.058 & 0.040 & 0.051 \\
}
{Výsledky pre funkciu Tangle, prvý riadok určuje aký bytecode bol použitý.}
{tab:exp2-tangle}



\resultsTable
{Výsledky pre funkciu Dynamic SHA}
{
	7 & 1.000\cc & 0.996\cc & 0.998\cc & 1.000\cc & 0.899\cc \\
	\hline
	8 & 1.000\cc & 1.000\cc & 1.000\cc & 1.000\cc & 0.631\cc \\
	\hline
	9 & 1.000\cc & 1.000\cc & 1.000\cc & 1.000\cc & 0.616\cc \\
}
{Výsledky pre funkciu Dynamic SHA, prvý riadok určuje aký bytecode bol použitý.}
{tab:exp2-dynamic-sha}

Výsledky z \hyperref[tab:exp2-dynamic-sha]{tabuľky 4.3} sú trochu prekvapivé, pretože EACirc je schopný rozlíšiť všetky rundy funkcie \textit{Dynamic SHA}, zatiaľ čo štatistické sady sú schopné rozoznať funkciu len po rundu 7. To isté platí aj pre EACirc s bežnými uzlami. Štatistické testy boli spustené nad dátami, ktoré vygeneroval generátor, ktorý obsahuje EACirc. Z toho vyplýva, že chybu generátora môžeme vylúčiť a funkcia buď nespĺňa niektorú z požiadaviek na náhodnosť, ktorú štatistické testy netestujú, alebo sa v implementácii EACircu nachádza chyba, ktorú sme prehliadli. 

\resultsTable
{Výsledky pre funkciu Dynamic SHA-2}
{
	10 & 1.000\cc & 1.000\cc & 1.000\cc & 1.000\cc & 1.000\cc \\
	\hline
	11 & 0.986\cc & 0.967\cc & 0.877\cc & 0.901\cc & 0.753\cc \\
	\hline
	12 & 0.059 & 0.056 & 0.042 & 0.057 & 0.053 \\
}
{Výsledky pre funkciu Dynamic SHA-2, prvý riadok určuje aký bytecode bol použitý.}
{tab:exp2-dynamic-sha-2}

\resultsTable
{Výsledky pre funkciu Decim}
{
	4 & 0.998\cc & 0.137\cc & 0.209\cc & 0.203\cc & 0.114\cc \\
	\hline
	5 & 0.802\cc & 0.073\cc & 0.095\cc & 0.079\cc & 0.053 \\
	\hline
	6 & 0.065 & 0.045 & 0.046 & 0.056 & 0.048 \\
}
{Výsledky pre funkciu Decim, prvý riadok určuje aký bytecode bol použitý.}
{tab:exp2-decim}

Výsledky z tabuliek \hyperref[tab:exp2-tangle]{4.2} až \hyperref[tab:exp2-decim]{4.5} sú trochu nejednoznačné. Z môjho pohľadu som v nich nenašiel žiadnu spojitosť medzi skúmanou funkciou a bytecodom z tej istej funkcie. Ba dokonca ani jeden z bytecodov sa nedá vyhlásiť za najlepší pre všetky funkcie. Na druhú stranu, výsledky v rámci funkcií sú celkom konzistentné. Pretože napríklad na funkcii \textit{Decim} vidno, že pre bytecody z funkcií \textit{Dynamic SHA} a \textit{Dynamic SHA-2} má veľmi podobné výsledky. Čo je očakávatelné, pretože implementácia týchto dvoch funkcií je veľmi podobná a teda ich bytecody obsahujú podobné inštrukcie.

Ďalšie pozitívum je, že na rozdiel od prvého experimentu dosiahli JVM uzly vo väčšine prípadoch porovnateľné výsledky ako bežné uzly. Najhoršie výsledky sme dosiahli v prípade funkcie Decim, preto  som sa rozhodol, že na tejto funkcii vyskúšam výpočty s viac generáciami. Počet generácií som zvolil na 300000, tým pádom bude každý beh EACircu vylepšovať výsledný obvod 10-krát dlhšie. Z dôvodu veľmi dlhého výpočtu, som musel znížiť počet behov z 1000 na 400. To znamená, že tieto výsledky by mali mať menšiu mieru významnosti oproti výpočtom, ktoré bežali až 1000 krát. Avšak pre vytvorenie obrazu o tom ako sa EACirc s JVM uzlami správa pri zvýšení počtu generácií by malo byť 400 behov dostačujúcich. 

\begin{table}[ht]
	
	\renewcommand{\arraystretch}{1.2}
	
	\begin{tabularx}{\textwidth}{|C||C|C|C|C|}
		\hline
		\multicolumn{5}{|c|}{\textbf{Výsledky pre Decim s použitím 300000 generácií}} \\
		\hline \hline
		\vspace{5pt}
		\textbf{Runda} &
		\vspace{0pt}
		\begin{tabular}[b]{@{}c}\small\textbf{Tangle} \\ \scriptsize(proportion) \end{tabular} &
		\vspace{-10pt}
		\begin{tabular}[b]{@{}c}\small\textbf{Dynamic} \\ \small\textbf{SHA} \\ \scriptsize(proportion) \end{tabular} &
		\vspace{-10pt}
		\begin{tabular}[b]{@{}c}\small\textbf{Dynamic} \\ \small\textbf{SHA-2} \\ \scriptsize(proportion) \end{tabular} &
		\vspace{0pt}
		\begin{tabular}[b]{@{}c}\small\textbf{Decim} \\ \scriptsize(proportion) \end{tabular} \\
		\hline\hline
		4 & 1.000\cc & 0.954\cc & 0.967\cc & 0.659\cc \\
		\hline
		5 & 0.669\cc & 0.453\cc & 0.465\cc & 0.320\cc \\
		\hline
		6 & 0.039 & 0.061 & 0.048 & 0.058 \\
		\hline
		
	\end{tabularx}
	\caption{Výsledky pre funkciu Decim s použitím 300000 generácií, prvý riadok určuje aký bytecode bol použitý.}
	\label{tab:exp2-decim-300k}
\end{table}

\hyperref[tab:exp2-decim-300k]{Tabuľka 4.6} neobsahuje výsledky pre bežné uzly, pretože výsledky pre bežné uzly s 300000 generáciami nemáme k dispozícii. Výsledky sa dajú porovnať s výsledkami z \hyperref[tab:exp2-decim]{tabuľky 4.5}, konkrétne stĺpec s označením bežné uzly. Treba však brať na vedomie, že bežné uzly mali na hľadanie obvodu 10-krát menej generácií. Čo sa týka výsledkov, JVM uzly sa znovu priblížili výsledkom, ktoré dosiahol EACirc s bežnými uzlami na 30000 generáciách. Avšak neprekonal žiadnu ďalšiu rundu. Znovu si môžeme všimnúť mieru konzistencie v rámci tejto funkcie. Napríklad spojenie medzi výsledkami pre bytecode z funkcií \textit{Tangle} a \textit{Decim}. Je vidieť, že pre rundu 4 je výpočet s bytecodom z funkcie \textit{Decim} približne o polovicu horší a to isté platí aj pre rundu 5. 


\section{Výpočty so zložitejším bytecodom}
\label{sec:exp3}

Ďalší a zároveň posledný experiment, ktorý som vyskúšal, bol postavený na myšlienke použitia bytecodu, ktorý obsahuje zložitejšie konštrukcie. Otázkou teda je, či bude genetika schopná využiť takéto konštrukcie na zhotovenie úspešnejších obvodov. Na vytvorenie bytecodu sme použili implementáciu šifrovacej funkcie AES~\parencite{AES-FIPS}.

Nastavenia pre tento experiment boli rovnaké, ako v predchádzajúcom experimente, teda 30000 generácií, 1000 testovacích vektorov a 1000 behov. Takisto som pre funkciu \textit{Decim} spustil výpočty aj pre verziu s 300000 generáciami a 400 behmi. 
	\begin{table}[ht]
\twoColumns{		
		\renewcommand{\arraystretch}{1.2}
		
		\begin{tabularx}{7.3cm}{|C||C|}
			\hline
			\multicolumn{2}{|c|}{\textbf{Tangle}} \\
			\hline \hline
	
			\textbf{Runda} &
	
			\textbf{Proportion} \\
			\hline\hline
			21 & 0.574\cc \\
			\hline
			22 & 0.578\cc \\
			\hline
			23 & 0.049  \\
			\hline
			
		\end{tabularx}
		\caption{Výsledky pre funkciu Tangle s použitím bytecodu z funckie AES.}
		\label{tab:exp3-tangle}
		
		\begin{tabularx}{7.3cm}{|C||C|}
			\hline
			\multicolumn{2}{|c|}{\textbf{Dynamic SHA-2}} \\
			\hline \hline
			
			\textbf{Runda} &
			
			\textbf{Proportion} \\
			\hline\hline
			10 & 1.000\cc \\
			\hline
			11 & 0.957\cc \\
			\hline
			12 & 0.056  \\
			\hline
			
		\end{tabularx}
		\caption{Výsledky pre funkciu Dynamic SHA-2 s použitím bytecodu z funckie AES.}
		\label{tab:exp3-dynamic-sha-2}
}{
		\renewcommand{\arraystretch}{1.2}
		
		\begin{tabularx}{7.3cm}{|C||C|}
			\hline
			\multicolumn{2}{|c|}{\textbf{Dynamic SHA}} \\
			\hline \hline
			
			\textbf{Runda} &
			
			\textbf{Proportion} \\
			\hline\hline
			7 & 0.977\cc \\
			\hline
			8 & 0.985\cc \\
			\hline
			9 & 0.992\cc  \\
			\hline
			
		\end{tabularx}
		\caption{Výsledky pre funkciu Dynamic SHA s použitím bytecodu z funckie AES.}
		\label{tab:exp3-dynamic-sha}
		
		\begin{tabularx}{7.3cm}{|C||C|}
			\hline
			\multicolumn{2}{|c|}{\textbf{Decim}} \\
			\hline \hline
			
			\textbf{Runda} &
			
			\textbf{Proportion} \\
			\hline\hline
			4 & 0.640\cc \\
			\hline
			5 & 0.187\cc \\
			\hline
			6 & 0.056  \\
			\hline
			
		\end{tabularx}
		\caption{Výsledky pre funkciu Decim s použitím bytecodu z funckie AES.}
		\label{tab:exp3-decim}
}

	\vspace{15pt}
	\begin{tabularx}{\textwidth}{|C||C|}
		\hline
		\multicolumn{2}{|c|}{\textbf{Decim - 300000 generácií}} \\
		
		\hline \hline
		
		\textbf{Runda} &
		
		\textbf{Proportion} \\
		\hline\hline
		4 & 1.000\cc \\
		\hline
		5 & 0.982\cc \\
		\hline
		6 & 0.072\cc  \\
		\hline
		
	\end{tabularx}
	\caption{Výsledky pre funkciu Decim s použitím bytecodu z funckie AES a počtom generácií 300000.}
	\label{tab:exp3-decim-300k}

\end{table}


Z tabuliek \hyperref[tab:exp3-tangle]{4.7} až \hyperref[tab:exp3-decim]{4.10} vyplýva, že EACirc s bytecodom z funkcie AES sa úspešnosťou vyrovnáva použitiu bytecodu z funkcií v predchádzajúcom experimente až na funkciu Decim, kde dosiahol výrazne lepších výsledkov, avšak stále sa nevyrovnáva úspešnosti bežných uzlov (\hyperref[tab:exp2-decim]{tabuľka 4.5}). Z toho vyplýva, že môžeme potvrdiť, že použitie zložitejšieho bytecodu je pre genetiku v niektorých prípadoch výhodnejšie ako použitie bytecodu z jednoduchších funkcií.

Vo výsledkoch v \hyperref[tab:exp3-decim-300k]{tabuľke 4.11} sa podarilo JVM uzlom prekonať EACirc s bežnými uzlami. Avšak bežné uzly boli spúšťané len s 30000 generáciami, takže mali na hľadanie výsledného obvodu 10 krát menej generácií. Dokonca sa nám podarilo prekonať bežné uzly o jednu rundu, aj keď iba veľmi tesne. Keďže sa jedná o výpočet v ktorom bolo spustených len 400 behov, pravdepodobne sa bude jednať len o štatistickú odchýlku. Jediným spôsobom ako to overiť je spustiť ďalšie behy z tohoto experimentu, avšak výpočet behov na 300000 generáciách je časovo veľmi náročný a preto som tieto výpočty už nespúšťal.





















